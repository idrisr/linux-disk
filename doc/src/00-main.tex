\documentclass[openany, 12pt]{book}
\input{preamble}
\title{Linux Disk}
\author{Idris}
\date{June 2025}

% chktex-file 1
\begin{document}
\tableofcontents

\chapter{Block Devices}

All disks are just devices—like keyboards or GPUs—managed by the Linux kernel.
What makes disks special is that they are block devices, meaning data is read
and written in fixed-size chunks with support for random access. This structure
enables filesystems, caching, and efficient IO.\@

\section{Block}
A block device enables:

\begin{center}
	\begin{tabular}{ll}
		\toprule
		caching      & blocks numbered---kernel can cache blocks in ram (page cache).    \\
		filesystems  & reading/writing data/metadata offsets---needs random block access \\
		efficient io & dag                                                               \\
		\bottomrule
	\end{tabular}
\end{center}

\begin{definition}{Source Of Truth}{}
	The same block can exist in memory or on disk, which begs the question---which
	is the source of truth? To write data, the block must be moved into memory,
	and then made ``dirty'', ie changed.  The dirty block is the source of
	truth.  For a clean block, the source of truth is memory or disk, though
	most blocks will not be in memory.
\end{definition}

\begin{definition}{drive firmware}{}
	The disk controller (in the drive firmware) exposes the disk as a Linear
	Block Address (LBA) space—a flat array of block numbers like:

	\begin{align*}
		\text{LBA}_0, \text{LBA}_1, \ldots, \text{LBA}_N
	\end{align*}

	When the OS issues a read/write to block 42, the controller translates that
	to a physical location (e.g., cylinder/head/sector or NAND page). This
	abstraction is standardized, which is why all block devices—from spinning
	HDDs to NVMe SSDs—present the same numbered interface.
\end{definition}

\begin{table}[h]
	\centering
	\begin{tabular}{lll}
		\toprule
		\textbf{Type}      & \textbf{Example}               & \textbf{Description}          \\
		\midrule
		Whole disk         & \texttt{/dev/sda}              & Entire physical disk          \\
		Partition          & \texttt{/dev/sda1}             & Subdivision of a disk         \\
		Loop device        & \texttt{/dev/loop0}            & File-backed block device      \\
		LVM logical volume & \texttt{/dev/mapper/vg-lv}     & Device-mapper virtual volume  \\
		Software RAID      & \texttt{/dev/md0}              & Aggregated disk via mdadm     \\
		ZRAM device        & \texttt{/dev/zram0}            & Compressed RAM-backed block   \\
		DM-Crypt (LUKS)    & \texttt{/dev/mapper/cryptroot} & Encrypted block device        \\
		RAM disk           & \texttt{/dev/ram0}             & Legacy RAM-based block device \\
		\bottomrule
	\end{tabular}
	\caption{Common Block Device Types in Linux}
\end{table}

\section{Sector}
\begin{definition}{sector}{}
	While the term block is the most commonly encountered one when talking about
	``block'' devices, disk devices are actually addressed by sector. A sector
	is the smallest addressable unit of a block device—typically \SI{512}{\byte}
	or \SI{4}{\kibi\byte}.

	Filesystems often operate on larger logical blocks built from sectors (e.g.,
	\SI{4}{\kibi\byte}), but the sector remains the atomic I/O unit for hardware
	and the kernel.

	Just learn to keep that tension in your brain as you read
	this, as that inexact phrasing dominates nearly all literature about this
	topic.
\end{definition}

\begin{table}[h]
	\centering
	\begin{tabular}{llll}
		\toprule
		\textbf{name} & \textbf{typical size}                   & \textbf{who uses it}                  & \textbf{what for}              \\
		\midrule
		page          & \SI{4}{\kibi\byte}                      & memory (MMU, kernel)                  & virtual memory, paging         \\
		sector        & \SI{512}{\byte}--\SI{4}{\kibi\byte}     & storage hardware, kernel              & atomic disk I/O                \\
		block         & \SI{4}{\kibi\byte}--\SI{64}{\kibi\byte} & filesystem                            & allocation, structure          \\
		cluster       & \SI{4}{\kibi\byte}--\SI{64}{\kibi\byte} & FAT/NTFS filesystems                  & group of blocks for allocation \\
		extent        & variable                                & modern filesystems (ext4, XFS, Btrfs) & contiguous allocation range    \\
		\bottomrule
	\end{tabular}
	\caption{Standard units in memory and storage}
\end{table}

\section{FileSystem}
\begin{table}[h]
	\setlength{\tabcolsep}{4pt}
	\centering
	\resizebox{\textwidth}{!}{%
		\begin{tabular}{llllllll}
			\toprule
			{filesystem} & {category}  & {allocation} & {metadata}      & {recovery}     & {snapshots}  & {checksums}    & {compression}   \\
			\midrule
			ext4         & local       & block bitmap & journaling      & journal        & no           & no             & no              \\
			xfs          & local       & extents      & b-tree + log    & log replay     & no           & metadata only  & no              \\
			btrfs        & local       & extents      & cow b-trees     & atomic cow     & yes          & yes            & yes             \\
			f2fs         & local       & segments     & nat/sit         & log replay     & no           & optional       & optional        \\
			zfs          & local       & extents      & cow b-trees     & trans.\ groups & yes          & yes            & yes             \\
			apfs         & local       & extents      & b-trees + cow   & atomic cow     & yes          & yes            & yes             \\
			ntfs         & local       & clusters     & mft             & journal        & shadow copy  & metadata only  & yes             \\
			fat32        & local       & clusters     & fat table       & none           & no           & no             & no              \\
			nfs          & network     & n/a          & server-side     & n/a            & server-side  & server-side    & server-side     \\
			sshfs        & network     & n/a          & passthrough     & n/a            & no           & no             & no              \\
			cephfs       & network     & object-based & distributed     & fs journal     & yes          & yes            & yes             \\
			fuse         & user-space  & n/a          & user-defined    & n/a            & user-defined & user-defined   & user-defined    \\
			overlayfs    & overlay     & passthrough  & union of dirs   & volatile       & upper layer  & no             & no              \\
			tmpfs        & memory      & ram pages    & kernel-internal & none           & no           & no             & yes (zram)      \\
			iso9660      & read-only   & fixed layout & static table    & none           & no           & optional (crc) & optional (rare) \\
			ipfs         & p2p/content & blockstore   & merkledag       & immutable      & yes          & yes (by hash)  & no              \\
			\bottomrule
		\end{tabular}
	}
	\caption{Comparison of filesystems by key characteristics and category}
\end{table}

\begin{definition}{allocation}{}
	How a filesystem decides where on disk to place file data. Some use
	fixed-size blocks, others group them into extents. Some delay allocation
	until writeback, others allocate immediately. The strategy affects
	fragmentation, performance, and metadata overhead.
\end{definition}

\begin{definition}{extent}{}
	Contiguous range of blocks on disk treated as a single unit. Instead of
	tracking each block individually, the filesystem records the starting block
	and the length. This reduces metadata overhead and improves performance for
	large files.
\end{definition}

\begin{enumerate}
	\item \textbf{Structure \& Organization}
	      \begin{enumerate}
		      \item \textbf{Filesystem:} The overall structure of how files and
		            directories are arranged on storage.
		      \item \textbf{Metadata:} Information *about* files, such as names,
		            sizes, timestamps, and permissions.
	      \end{enumerate}
	\item \textbf{Data Management \& Efficiency}
	      \begin{enumerate}
		      \item \textbf{Allocation:} The method by which disk space is
		            assigned to files.
		      \item \textbf{Compression:} Reducing file size to optimize storage
		            space and transfer speeds.
	      \end{enumerate}
	\item \textbf{Resilience \& Advanced Features}
	      \begin{enumerate}
		      \item \textbf{Recovery:} The ability to restore the file system to
		            a consistent state following failures or data corruption.
		      \item \textbf{Checksums:} Mechanisms for verifying data integrity,
		            detecting accidental changes or corruption.
		      \item \textbf{Snapshots:} Point-in-time copies of the file system,
		            crucial for backups, versioning, and disaster recovery.
	      \end{enumerate}
\end{enumerate}

\begin{definition}{Filesystem}{}
	A \textbf{filesystem} is a kernel-resident function that maps
	\textit{inode-level operations} to \textit{block-level operations}: \[
		\text{inode\_op} \;\Rightarrow\; \text{[block\_op]} \] It translates
	high-level actions like \texttt{open}, \texttt{read}, \texttt{write}, and
	\texttt{mkdir} into precise sequences of reads and writes on a block device.

	Once mounted, the filesystem hooks into the kernel's VFS tree, and all
	accesses under the mount point are routed through this translation layer.
\end{definition}

\chapter{Device Mapper}
\begin{align*}
	I         & := \text{inode ops}       \\
	S         & := \text{sector ops}      \\
	f         & := \text{file system ops} \\
	d         & := \text{device mapper}   \\
	f         & :  I \to S                \\
	d         & :  S \to S                \\
	d \circ f & : I \to S                 \\
\end{align*}

\begin{definition}{luks}{}
	\begin{align*}
		S_1 & := \text{plain sector}     \\
		S_2 & := \text{encrypted sector} \\
		d   & := \text{luks}             \\
		d   & : S_1 \to S_2
	\end{align*}
	\begin{alist}
		\item transparent sector-level encryption
		\item decryption happens on io path
		\item user never sees ciphertext; the disk never sees plaintext.
		\item implemented via \texttt{dm-crypt} and \texttt{cryptsetup}
	\end{alist}
\end{definition}

\begin{definition}{pv (physical volume)}{}
	\begin{align*}
		S_0 & := \text{raw sector device}        \\
		S_1 & := \text{lvm-tracked sector space} \\
		d   & := \text{pv init}                  \\
		d   & : S_0 \to S_1
	\end{align*}
	\begin{alist}
		\item first wedge in the LVM stack
		\item just marks a sector as eligible for lvm
		\item stores metadata in reserved sectors
		\item created via \texttt{pvcreate}
	\end{alist}
\end{definition}

\chapter{Disko}
\begin{definition}{Purpose}{}
	The purpose of disko is to go from a raw block storage device to a fully
	configured, partitioned disk with installed filesystems, ready to mount or
	ready for boot.
\end{definition}

device
disk
partition

\chapter{Tools}
\begin{definition}{fdisk}{}
	dialogue-driven command-line utility that creates and manipulates partition
	tables and partitions on a hard disk. Hard disks are divided into partitions
	and this division is described in the partition table.
	\begin{verbatim}
fdisk -l <block device>
    \end{verbatim}
\end{definition}

\chapter{LUKS}
\begin{definition}{LUKS}{}
	linux unified key setup
\end{definition}

\begin{intuition}{LUKS}{}
	It's a way to encrypt block level disks at rest. Once you decrypt and mount
	it, it appears just like any other device or filesystem for an application
	that is using it.
\end{intuition}

\begin{enumerate}[label = {(\arabic*)}]
	\item dm-crypt
	\item full disk encryption
\end{enumerate}

\begin{verbatim}
sudo cryptsetup luksOpen /dev/sda4 my-drive
sudo mkfs.ext4 /dev/mapper/my-drive -L my-drive
sudo mkdir /mnt/data
sudo mount /dev/mapper/my-drive /mnt/data
sudo cryptsetup luksAddKey --key-slot 1
sudo cryptsetup luksHeaderRestore /dev/sda4
cryptsetup luksDump /dev/xvda5
cryptsetup luksHeaderBackup /dev/xvda5 --header-backup-file LuckyHeader.bin
cfdisk /dev/xvdb
cryptsetup luksFormat /dev/xvdb
cryptsetup luksOpen /dev/xvd DataDrive
mkfs.ext 4 /dev/mapper/DataDrive
fsck /dev/mapper/DataDrive
\end{verbatim}

\begin{verbatim}
lsblk
fdisk /dev/sdd
cryptsetup open /dev/sdd1 drive
mount /dev/mapper/drive /mnt
cryptsetup close drive
\end{verbatim}

\begin{definition}{dm-crypt}{}
	a linux device mapper and kernel module for encryption
\end{definition}

\begin{definition}{device mapper}{}
	Framework provided by the Linux kernel for mapping physical block devices
	onto higher-level virtual block devices. It forms the foundation of the
	\begin{enumerate}[label = {(\arabic*)}]
		\item logical volume manager (LVM)
		\item software RAIDs
		\item dm-crypt disk encryption
		\item file system snapshots
	\end{enumerate}
	\begin{align*}
		\text{deviceMapper}: \text{block} \to \text{block}
	\end{align*}
\end{definition}

\begin{enumerate}[label = {(\arabic*)}]
	\item pvcreate
	\item vgcreate
	\item lvcreate
\end{enumerate}

cryptsetup
keyslot
luks header

\end{document}
